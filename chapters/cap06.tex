\chapter{Conclusions and Future Work}\label{cap:conclusions}

In conclusion, this thesis has presented a comprehensive analysis and development of a system for efficient detection and measurement of wood leftovers in the manufacturing industry. Through the integration of image processing techniques and statistical analysis, the system has demonstrated its ability to accurately detect and measure the dimensions of wood leftovers.

The results obtained from the performance evaluation of the system have shown promising outcomes. The comparison between manual measurements and measurements obtained through the system indicated a strong correlation and a small systematic error. This suggests that the developed system can be a viable alternative to replace the traditional manual measurement method, offering greater efficiency and reducing human effort.

Furthermore, the system incorporates robust security measures to protect against potential cyber threats. The implementation of techniques such as code injection prevention, denial-of-service mitigation, and access control based on role-based authentication ensures the integrity, availability, and confidentiality of the data stored in the system.

The ability to store and share data related to projects and leftovers has been successfully implemented. The system provides a user-friendly interface that allows users to access and manage project information, share media files, and collaborate effectively. The integration of the system with the company's database ensures the storage and analysis of relevant data, facilitating informed decision-making and project management.

In terms of limitations, there were challenges encountered during the development of the architecture due to the instability of the Fiware platform across different versions. Furthermore, the system requires improvements in various aspects, such as systematic testing to avoid errors in future modifications and additional quality assurance testing.

For future work, the utilization of artificial intelligence techniques, specifically object detection, could be explored to automate the wood selection process entirely. Although attempts were made in this direction in the present work, the lack of a sufficient number of reliable wood leftover images from the industry restricted the presentation of conclusive results. However, this remains a possibility for future research and development.