%\thispagestyle{empty}


No contexto contemporâneo, marcado por uma competição global intensificada e pela constante evolução do cenário de globalização, torna-se imperativo para as indústrias, incluindo as Pequenas e Médias Empresas (PMEs), empreender esforços para aprimorar seus processos operacionais, frequentemente pela via da adaptação tecnológica digital. O presente estudo insere-se dentro do escopo do projeto denominado “Wood Work 4.0”, cujo propósito é infundir inovação na indústria de fabricação de móveis de madeira por meio da otimização de processos e da adoção de tecnologias digitais. Este projeto obteve financiamento do Fundo de Desenvolvimento da União Europeia, em colaboração com o programa Regional do Norte 2020 e foi realizado na empresa Carpintaria Mofreita, localizada em Macedo de Cavaleiros, Portugal. Nesse sentido, este estudo introduz uma arquitetura de software que oferece suporte à rastreabilidade de projetos na indústria de móveis de madeira, e simultaneamente emprega um sistema para identificar e gerenciar sobras de material, objetivando uma gestão de resíduos mais eficiente. Para o desenvolvimento dessa arquitetura de software, adotou-se uma abordagem que integra a plataforma Fiware, especializada em sistemas para a Internet das Coisas (IoT), com uma Interface de Programação de Aplicações (API) criada especificamente para gerenciar informações de usuários, projetos, e arquivos de mídia associados. O sistema de identificação de sobras de material emprega técnicas de processamento de imagem para extrair características geométricas dos materiais. Adicionalmente, esses dados são integrados ao banco de dados da empresa. Desta forma, foi possível desenvolver uma arquitetura que permite não só capturar informações de projetos, mas também gerenciá-las de forma eficaz. No caso da identificação de sobras de material, o sistema foi capaz de estabelecer, com um grau de precisão satisfatório, as dimensões dos materiais, possibilitando a inserção desses dados no banco de dados da empresa para gestão e otimização do uso de recursos.

\bigskip

\noindent \textbf{Palavras-chave:} Projeto Wood Work 4.0 ;  Internet das Coisas; Processamento de Imagem; Digitalização Industrial.
