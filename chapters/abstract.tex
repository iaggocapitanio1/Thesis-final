%\thispagestyle{empty}


In the contemporary context, characterized by intensified global competition and the constant evolution of the globalization landscape, it becomes imperative for industries, including Small and Medium Enterprises (SMEs), to undertake efforts to enhance their operational processes, often through digital technological adaptation. The present study falls within the scope of the project named “Wood Work 4.0,” which aims to infuse innovation into the wood furniture manufacturing industry through process optimization and the adoption of digital technologies. This project received funding from the European Union Development Fund, in collaboration with the North 2020 Regional Program, and was carried out at the Carpintaria Mofreita company, located in Macedo de Cavaleiros, Portugal. In this regard, this study introduces a software architecture that supports the traceability of projects in the wood furniture industry and simultaneously employs a system to identify and manage material leftovers, aiming for more efficient waste management. For the development of this software architecture, an approach that integrates the Fiware platform, specialized in systems for the Internet of Things (IoT), with an Application Programming Interface (API) specifically created to manage information about users, projects, and associated media files, was adopted. The material leftovers identification system employs image processing techniques to extract geometric characteristics of the materials. Additionally, these data are integrated into the company’s database. In this way, it was possible to develop an architecture that allows not only the capturing of project information but also its effective management. In the case of material leftovers identification, the system was able to establish, with a satisfactory degree of accuracy, the dimensions of the materials, enabling the insertion of these data into the company's database for resource management and optimization.

\bigskip

\noindent \textbf{Keywords:} Wood Work 4.0 Project; Internet of Things ; Image Processing; Industrial Digitalization.

