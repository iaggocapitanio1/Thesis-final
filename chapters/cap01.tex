\chapter{Introduction}\label{cap:intro}

This thesis is directly correlated with the context of the Wood Work 4.0 project, funded by the European Union's Regional Development Fund in conjunction with the NORTE 2020 regional investment program and promoted by Mofreita Industries. The case study in this research is conducted at Mofreita Industries, a company engaged in the production of furniture in the wood industry sector.

In this company, despite significant advancements in modern machinery, the operational processes are still carried out in an outdated manner. Many project management and tracking operations are performed using paper-based methods and other outdated techniques, negatively impacting the efficiency of the system. Hence, the primary objective of this project is to foster enhanced digitalization within the wood industry, specifically focusing on the domain of furniture manufacturing.

Furthermore, the wood furniture manufacturing industry faces a recurring challenge in effectively managing cut-offs or remnants. Two distinct categories of cut-offs exist: those suitable for reuse and those that lack any potential for re-purposing. The non-reusable cut-offs encompass damaged wooden pieces or those with significantly limited cutting areas. Conversely, reusable cut-offs possess ample usable space. Unfortunately, these reusable remnants are not adequately documented or cataloged; instead, they are haphazardly stacked within storage areas, devoid of any systematic inventory management practices. This inadequacy gives rise to two significant complications: first, the inefficient utilization of available pieces, as workers may opt to use larger wood sections instead of selecting more suitable alternatives, and second, the wastage of valuable human resources and time spent on the laborious task of searching for an appropriate piece to fulfill specific project requirements.

Indeed, in the contemporary business scenario, there is a relentless effort by companies to optimize their production processes through the use of new technologies \cite{JewapatarakulDigitalTransformation, DingEffectsofIoT, GUNTHER2017191}. The declining competitiveness of the industrial sector in European countries has led to the adoption of an industrial strategy driven by the use of technologies such as robotics, sensors, Big Data, machine learning, and telecommunication networks between devices \cite{HerreroMeasuringTheEffectivenessOfIndustrialProcesses}. This new approach to industrialization, encouraged by the European Union (EU) countries, aims to improve the utilization of natural resources and increase the competitiveness of industries. This approach uses a deeper integration of the manufacturing processes, with technologies that allow rapid sharing of data between production processes. In this way, value and information can be added to the production chain, increasing competitiveness\cite{Grabowska+2020+90+96}.

This new paradigm implies a broader digitization of the industry, transcending simple management and resulting in the creation of \emph{"smart products"} that are integrated into the value chain and are part of the company's virtual environment. This allows tracking the location of these products in the manufacturing environment, identification of the production process they are in, and provision of information to customers about the status of their orders \cite{economies6030046}. These possibilities contribute to the improvement of competitiveness and the integration of processes \cite{SeungSME}.

Building upon these advancements, the main idea of Industry 4.0 is to create a corporate network that allows data exchange between intelligent components capable of communicating with each other \cite{Grabowska+2020+90+96}. In this context, we highlight the \acrfull{iiot}\footnote{Industrial Internet of Things is a computing concept that describes communication between devices and services on an enterprise network \cite{Sisinni8401919}.}, a technology model that fits perfectly in this  of data sharing dynamics \cite{GARG2022286}.


\acrfull{iiot}, a technology that enables task and process automation, operates through a \acrfull{m2m} type communication. This form of communication is characterized by the exchange of data between devices and services, without the need for human intervention. This process is enabled by the use of sensors, \acrfull{rfid} tags, network services, and other devices that are connected to the Internet. These devices can not only communicate with each other, but also interact with devices external to the network\cite{KhanIoT}. Therefore, through \acrshort{m2m} communication, companies can automate tasks and processes, allowing devices and services to collect real-time data and make decisions autonomously and instantaneously. For example, \acrshort{m2m} communication can be used to monitor the location of parts in the production environment, and reporting productions delays and unusul events. In this way, it is possible to improve the effectiveness of the manufacturing process, reduce waste and costs \cite{SeungSME}. This technology also can help corporations improve the user experience, both for the end consumer and the product development team \cite{Grabowska+2020+90+96}.

From the end user's point of view, M2M communication can enable real-time tracking of the development status of their customized product. This is possible by collecting real-time data about the production process and making this data available to the customer via a portal or application. Such a practice can give the customer a clearer view of the progress of his order, thereby increasing his satisfaction with the service.

From the product development team's point of view, M2M communication can allow individuals to quickly obtain information about previous products with similar characteristics to the product being developed. This information can be gathered through attributes such as geometric information, material, lead time, and more. This can help the team better understand customer needs and preferences, and speed up the modeling process.

Building upon the advantages of \acrfull{it} in the industry, the adoption of IT-driven approaches brings forth a multitude of benefits. By harnessing IT solutions, companies can gain a comprehensive understanding of their manufacturing operations, thereby delving deeper into both the products themselves and the desires of their customers. As a result, IT-driven approaches positively impact the entire product value chain, encompassing critical aspects like design, production, distribution, and customer satisfaction \cite{economies6030046}. Through the integration of IT, businesses can optimize operational efficiency, streamline workflows, enhance quality control, facilitate data-driven decision-making, and foster innovation in their respective industries. These advancements collectively contribute to heightened competitiveness, improved customer experiences, and overall business success.

The implementation of Industry 4.0 technologies in the furniture industry is still little explored, making this study of paramount importance. The project "Development of Traceability Solution for Furniture Components" uses image processing methods in order to reduce process losses in the furniture industry, aligning with goal number 12 of the UN's SDGs (Sustainable Development Goals), which seeks to promote sustainable production and consumption patterns. This traceability solution allows the tracking and monitoring of components used in furniture manufacturing, contributing to the promotion of sustainability throughout the production chain. The combination of the implementation of SDG 12 with image processing techniques in this project exemplifies the innovative application of technology, driving sustainability in the furniture industry and contributing to a more responsible and conscious future.

The issue of digitization in small and medium-sized enterprises (SMEs), particularly those involved in the secondary sector of wood processing, is a pressing matter that requires attention. Among the initiatives aimed at addressing this challenge, the Wood Work 4.0 project stands out as a significant endeavor funded by the NORTE 2020 program and the European regional development fund. This project brings together various partners, each with their own objectives, to explore the implementation of Industry 4.0 technologies in the furniture industry.

This theses was developed in the context of the Wood Work 4.0 project. It recognizes the limited exploration of Industry 4.0 technologies in the furniture industry, highlighting the paramount importance of this study. By utilizing image processing methods, the project aims to minimize process losses within the furniture industry, aligning its objectives with goal number 12 of the UN's Sustainable Development Goals (SDGs). This goal focuses on promoting sustainable production and consumption patterns, a crucial aspect in today's world.

The traceability architecture solution developed in this project enables the tracking and monitoring of components used in furniture manufacturing, thereby contributing to the promotion of sustainability throughout the entire production chain. By combining the implementation of SDG 12 with innovative image processing techniques, this project exemplifies the forward-thinking application of technology in driving sustainability within the furniture industry. Ultimately, the Woodwork 4.0 project aspires to create a more responsible and conscious future by fostering sustainable practices and leveraging the potential of digitalization in \acrshort{smes} involved in wood processing.

By addressing this gap, this study has the potential to contribute to the development of the furniture manufacturing sector. It can identify opportunities for improvement, develop innovative solutions, and drive digital transformation in the sector. Furthermore, this study can provide insights into how Industry 4.0 can be translated to a particular case study. This is especially relevant in an increasingly competitive and constantly evolving market, where companies must continuously seek innovations to stand out \cite{ghobakhloo2021industry}.


\section{Motivation}\label{cap:intro:justification}

The motivation behind this work is to promote technological advances in the wooden furniture manufacturing industry, making it more competitive on the world stage. The proposal consists in apply advanced image processing techniques to improve the management of wood waste generated during the production process. Currently, these surplus materials are often mismanaged and underutilized.

Implementing a waste management program based on image processing has the potential  to significantly reduce costs, optimizing the use of this waste contributes to the preservation of the environment by avoiding waste of natural resources and minimizing the need to cut down new trees for wood production. This reflects a more sustainable and ecologically conscious approach. In addition to bringing a positive image to the company.

Therefore, the ultimate goal of this work is to provide a technological solution that not only improves the wood waste management of the Mofreitas company, but also promotes a more sustainable practice within the furniture industrial sector. The implementation of this project represents an opportunity to move towards an efficient and responsible management of resources, benefiting both the company and the environment.


\section{Objectives}\label{cap:intro:objectives}

The main purpose of this thesis is to develop and implement an efficient strategy to manage, identify and reuse leftover material, particularly wood boards, in the wood work industry. To achieve this end, the following specific objectives have been outlined:

\begin{enumerate}
    \item \textbf{Develop a System for Storage and Identification of Scraps:} The goal is to create a robust system capable of properly storing the information of the remnant materials and accurately identifying them. This system should be able to categorize the materials based on various parameters, such as type, size, and quantity, facilitating their future reuse.
    
    \item \textbf{Develop Methodologies for Automatically Detecting the Geometry of Leftovers:} The development of methodologies capable of automatically detecting the geometry of the leftovers is the second goal. This includes creating algorithms and procedures that can accurately identify the dimensions and shape of the remnant materials, making them more easily reusable.
    
    \item \textbf{Integration of the Remnant Data with the Company Materials Database:} Another crucial goal is to integrate the leftover information with the company's existing database. This means incorporating the surplus data into the overall materials database, allowing a complete and up-to-date view of the resources available for production.
    
    \item \textbf{Develop an architecture that enables traceability of a piece of furniture:} The fourth objective is to establish a traceability system for each piece of furniture produced. This implies developing methods to track each piece from the beginning of production until the final delivery, providing transparency and total control over the production process.
    
    \item \textbf{Publication of Scientific Article:} Finally, it is the goal of this thesis to produce at least one scientific article for publication in a SCOPUS or ISI indexed journal or conference. This article will detail the results of the study and contribute to the existing body of knowledge in the field of materials management in the manufacturing industry.
\end{enumerate}

Each of these objectives contributes to the larger goal of creating an efficient leftover materials management system that can be implemented in an actual case study, with the potential to significantly improve the efficiency and sustainability of production processes.
